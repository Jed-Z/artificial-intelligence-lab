\documentclass[a4paper, 11pt]{article}
\usepackage{amsmath}
\usepackage{graphicx}
\usepackage{geometry}
\usepackage{listings}
\geometry{scale=0.8}
\linespread{1.5}
\usepackage{hyperref}
\usepackage{listings}
\usepackage{color}
\definecolor{dkgreen}{rgb}{0,0.6,0}
\definecolor{gray}{rgb}{0.5,0.5,0.5}
\definecolor{mauve}{rgb}{0.58,0,0.82}
\lstset{frame=shadowbox,
    language=Prolog,
    aboveskip=3mm,
    belowskip=3mm,
    showstringspaces=false,
    columns=flexible,
    basicstyle={\small\ttfamily},
    keywordstyle=\color{blue},
    commentstyle=\color{dkgreen},
    stringstyle=\color{mauve},
    breaklines=true,
    breakatwhitespace=true,
    tabsize=3
}

\title{	
\normalfont \normalsize
\textsc{School of Data and Computer Science, Sun Yat-sen University} \\ [25pt] %textsc small capital letters
\rule{\textwidth}{0.5pt} \\[0.4cm] % Thin top horizontal rule
\huge  E06 Queries on KB \\ % The assignment title
\rule{\textwidth}{2pt} \\[0.5cm] % Thick bottom horizontal rule
\author{17341203 Yixin Zhang}
\date{\normalsize\today}
}

\begin{document}
\maketitle
\tableofcontents
\newpage


\section{Problem Description}
Given a KB \texttt{Restaurants.pl}, which describes the distribution of branches of 10 well-known restaurants in Guangzhou. 

For example, \texttt{restaurant(ajukejiacai,2007,yuecai)} means that \texttt{ajukejiacai} was founded in 2007 and is a restaurant of \texttt{yuecai}. \texttt{branch(ajukejiacai,xintiandi)} means that \texttt{ajukejiacai} has a branch in \texttt{xintiandi}. \texttt{district(xintiandi,panyu)} means that \texttt{xintiandi} is an area of \texttt{panyu} district.

Please formulate each of the following questions as a query using Prolog's notation, pose it to Prolog, and obtain Prolog's answer:
\begin{enumerate}
  \item What restaurants have branches in beigang? 
  \item What districts have restaurants of yuecai and xiangcai?
  \item What restaurants have the least number of branches?
  \item What areas have two or more restaurants?
\item Which restaurant has the longest history?
\item What restaurants have at least 10 branches?
\end{enumerate}
Please define the new relation below using Prolog and test it.
\begin{itemize}
\item sameDistrict(Restaurant1, Restaurant2): Restaurant1 and Restaurant2 have one or more branches in the same district.
\end{itemize}




You should write down a listing that shows the queries you submitted to Prolog, and the answer returned. Hand in a file named \textsf{E06\_YourNumber.pdf}, and send it to \textsf{ai\_201901@foxmail.com}


\section{Codes and Results}
\subsection{Six Queries}

\begin{enumerate}
\item What restaurants have branches in beigang? 
\begin{lstlisting}
?- setof(X, branch(X,beigang), Result).
Result = [huangmenjimifan, mixuebingcheng, shaxianxiaochi].
\end{lstlisting}


\item What districts have restaurants of yuecai and xiangcai?

For convenience, I define a new auxiliary rule `categoryDistrict' which gives names of districts where there is at least one restaurant offering food of a specific category. To do such thing, `assertz/1' is used. Then this auxiliary rule is used to find the districts that have both yuecai and xiangcai.

Note that in my code `R' means restaurants, `C' means categories, `A' means areas, `D' means districts.
\begin{lstlisting}
?- assertz((categoryDistrict(C,D):-restaurant(R,_,C),branch(R,A), district(A,D))).
true.

?- setof(D, (categoryDistrict(yuecai,D),categoryDistrict(xiangcai,D)), Result).
Result = [haizhu, liwan, panyu, tianhe, yuexiu].
\end{lstlisting}


\item What restaurants have the least number of branches?
\begin{enumerate}
    \item Step 1: Find branches of each restaurant
    \begin{lstlisting}
?- setof(A,branch(R,A),As).
R = ajukejiacai,
As = [shatainan, xintiandi, yongfu] ;
R = dagangxianmiaoshaoji,
As = [beishan, cencun, changxing, dongpu, fangcun, gaosheng, huadong, kecun, nanpudadao|...] ;
R = diandude,
As = [bainaohui, huachengdadao, huifudong, linhe, panfu, shiqiao, tianhebei, yangji, youtuobangshiguang|...] ;
R = hongmenyan,
As = [xintiandi, zhilanwan] ;
... % omitted
    \end{lstlisting}
    
    \item Step 2: Count their branches
    \begin{lstlisting}
?- setof(
|        Count,
|        As^( restaurant(R,_,_), setof(A,branch(R,A),As), length(As,Count)),
|        Counts
|    ).
R = ajukejiacai,
Counts = [3] ;
R = dagangxianmiaoshaoji,
Counts = [11] ;
R = diandude,
Counts = [10] ;
R = hongmenyan,
Counts = [2] ;
R = huangmenjimifan,
Counts = [7] ;
R = mixuebingcheng,
Counts = [12] ;
R = muwushaokao,
Counts = [10] ;
R = shaxianxiaochi,
Counts = [3] ;
R = tongxianghui,
Counts = [10] ;
R = yangguofu,
Counts = [4].
    \end{lstlisting}
    
    \item Step 3: Make a list of counts
    \begin{lstlisting}
?- findall(
|        Counts,
|        setof(
|            Count,
|            As^( restaurant(R,_,_), setof(A,branch(R,A),As), length(As,Count)),
|            Counts
|        ),
|        CountList
|    ).
CountList = [[3], [11], [10], [2], [7], [12], [10], [3], [10],[4]].
    \end{lstlisting}
    
    \item Step 4: Find the minimum value in the list
    \begin{lstlisting}
min_list(CountList, MinCount).
    \end{lstlisting}
    
    \item Step 5: Find the restaurant whose branch count equals to the minimum value
    \begin{lstlisting}
?- setof(
|        R,
|        Count^As^CountList^(
|            findall(
|                Counts,
|                setof(
|                    Count,
|                    As^( restaurant(R,_,_), setof(A,branch(R,A),As), length(As,Count)),
|                    Counts
|                ),
|                CountList
|            ),
|            min_list(CountList, MinCount), setof(A,branch(R,A),As), length(As,Count), Count = MinCount
|            ),
|            Result
|    ).
MinCount = 2,
Result = [hongmenyan].
    \end{lstlisting}
    
    \item Result: The result is hongmenyan. It has only 2 branches.
\end{enumerate}


\item What areas have two or more restaurants?

We use nested `setof' to accomplish this. The inner one gives those restaurants (`Rs') which are in area `A'; Then the outer one counts the restaurants correspoding to different areas, and select those areas with a quantity greater than 2. To count the number of elements in a list, function `length/2' is used.
\begin{lstlisting}
?- setof(
|        A,
|        Count^ Rs^ (
|            setof(R, branch(R,A), Rs),
|            length(Rs, Count), Count>=2
|        ),
|        Result
|    ).
Result = [bainaohui, beigang, dongpu, shiqiao, tianhebei, xintiandi, yongfu, yuancun].
\end{lstlisting}

\item Which restaurant has the longest history?

The idea is similar to Question 3, but this one is much easier. No more explanations needed.
\begin{lstlisting}
?- setof(
|        R,
|        YearList^Y^(
|            findall(
|                Ys,
|                setof(Y,C^restaurant(R,Y,C),Ys),
|                YearList
|            ), min_list(YearList, MinYear), restaurant(R,Y,_), Y = MinYear
|        ),
|        Result
|    ).
MinYear = 1935,
Result = [huangmenjimifan].
\end{lstlisting}


\item What restaurants have at least 10 branches?

The idea is quite similar to Question 4.
\begin{lstlisting}
?- setof(
|        R,
|        Count^ As^ (
|            setof(A, (branch(R,A)), As),
|            length(As, Count), Count>=10
|        ),
|        Result
|    ).
Result = [dagangxianmiaoshaoji, diandude, mixuebingcheng, muwushaokao, tongxianghui].
\end{lstlisting}

\end{enumerate}

\subsection{Same District Relation}

Define the same district rule as below.
\begin{lstlisting}
sameDistrict(R1,R2):-branch(R1,A1),branch(R2,A2),district(A1,D),district(A2,D),R1\=R2.
\end{lstlisting}

Test it:
\begin{lstlisting}
?- sameDistrict(hongmenyan,ajukejiacai).
true .

?- sameDistrict(muwushaokao,diandude).
true .
\end{lstlisting}

We can also print all the restaurant pairs where they have one or more branches in the same district.
\begin{lstlisting}
assertz((sameDistrict(R1,R2):-branch(R1,A1),branch(R2,A2),district(A1,D),district(A2,D),R1\=R2)).
assertz((printList([]))).
assertz((printList([H|T]):-write(H),nl,printList(T))).

?- setof((R1,R2), sameDistrict(R1,R2), Result), printList(Result).
ajukejiacai,dagangxianmiaoshaoji
ajukejiacai,diandude
ajukejiacai,hongmenyan
ajukejiacai,huangmenjimifan
ajukejiacai,mixuebingcheng
ajukejiacai,muwushaokao
ajukejiacai,shaxianxiaochi
ajukejiacai,tongxianghui
ajukejiacai,yangguofu
dagangxianmiaoshaoji,ajukejiacai
dagangxianmiaoshaoji,diandude
dagangxianmiaoshaoji,hongmenyan
dagangxianmiaoshaoji,huangmenjimifan
dagangxianmiaoshaoji,mixuebingcheng
dagangxianmiaoshaoji,muwushaokao
dagangxianmiaoshaoji,shaxianxiaochi
dagangxianmiaoshaoji,tongxianghui
dagangxianmiaoshaoji,yangguofu
diandude,ajukejiacai
diandude,dagangxianmiaoshaoji
diandude,hongmenyan
diandude,huangmenjimifan
...
% too much, omitted
\end{lstlisting}

\section{Experience}
When solving Qestion 2, I met a problem.

For example, I want to find out which districts have restaurants of yuecai. I tried the code below, but the result it produced was strange.
\begin{lstlisting}
?- setof(D, R^A^(restaurant(R,_,yuecai), branch(R,A), district(A,D)), Result).
Result = [panyu, tianhe, yuexiu] ;
Result = [baiyun, panyu, yuexiu] ;
Result = [baiyun, haizhu, huadu, liwan, panyu, tianhe].
\end{lstlisting}

Why three different lists? However, the correct code is:
\begin{lstlisting}
?- setof(D, R^Y^A^(restaurant(R,Y,yuecai), branch(R,A), district(A,D)), Result).
Result = [baiyun, haizhu, huadu, liwan, panyu, tianhe, yuexiu].
\end{lstlisting}
Note that the variable `Y' cannot be replaced by the anonymous variable `\_'. Otherwise the result will be three lists instead of one, because there are restaurants founded in there different years having both yuecai and xiangcai, grouped by area `A' and restaurant `R'. Thanks our TAs to help me understand this trick!

%\clearpage
%\bibliography{E:/Papers/LiuLab}
%\bibliographystyle{apalike}
\end{document} 
%%% Local Variables:
%%% mode: latex
%%% TeX-master: t
%%% End:
