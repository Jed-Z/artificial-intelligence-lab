
    




    
\documentclass[11pt]{article}
\usepackage[UTF8, scheme=plain, punct=plain, zihao=false]{ctex}
    
    \usepackage[breakable]{tcolorbox}
    \tcbset{nobeforeafter} % prevents tcolorboxes being placing in paragraphs
    \usepackage{float}
    \floatplacement{figure}{H} % forces figures to be placed at the correct location
    
    \usepackage[T1]{fontenc}
    % Nicer default font (+ math font) than Computer Modern for most use cases
    \usepackage{mathpazo}

    % Basic figure setup, for now with no caption control since it's done
    % automatically by Pandoc (which extracts ![](path) syntax from Markdown).
    \usepackage{graphicx}
    % We will generate all images so they have a width \maxwidth. This means
    % that they will get their normal width if they fit onto the page, but
    % are scaled down if they would overflow the margins.
    \makeatletter
    \def\maxwidth{\ifdim\Gin@nat@width>\linewidth\linewidth
    \else\Gin@nat@width\fi}
    \makeatother
    \let\Oldincludegraphics\includegraphics
    % Set max figure width to be 80% of text width, for now hardcoded.
    \renewcommand{\includegraphics}[1]{\Oldincludegraphics[width=.8\maxwidth]{#1}}
    % Ensure that by default, figures have no caption (until we provide a
    % proper Figure object with a Caption API and a way to capture that
    % in the conversion process - todo).
    \usepackage{caption}
    \DeclareCaptionLabelFormat{nolabel}{}
    \captionsetup{labelformat=nolabel}

    \usepackage{adjustbox} % Used to constrain images to a maximum size 
    \usepackage{xcolor} % Allow colors to be defined
    \usepackage{enumerate} % Needed for markdown enumerations to work
    \usepackage{geometry} % Used to adjust the document margins
    \usepackage{amsmath} % Equations
    \usepackage{amssymb} % Equations
    \usepackage{textcomp} % defines textquotesingle
    % Hack from http://tex.stackexchange.com/a/47451/13684:
    \AtBeginDocument{%
        \def\PYZsq{\textquotesingle}% Upright quotes in Pygmentized code
    }
    \usepackage{upquote} % Upright quotes for verbatim code
    \usepackage{eurosym} % defines \euro
    \usepackage[mathletters]{ucs} % Extended unicode (utf-8) support
    \usepackage[utf8x]{inputenc} % Allow utf-8 characters in the tex document
    \usepackage{fancyvrb} % verbatim replacement that allows latex
    \usepackage{grffile} % extends the file name processing of package graphics 
                         % to support a larger range 
    % The hyperref package gives us a pdf with properly built
    % internal navigation ('pdf bookmarks' for the table of contents,
    % internal cross-reference links, web links for URLs, etc.)
    \usepackage{hyperref}
    \usepackage{longtable} % longtable support required by pandoc >1.10
    \usepackage{booktabs}  % table support for pandoc > 1.12.2
    \usepackage[inline]{enumitem} % IRkernel/repr support (it uses the enumerate* environment)
    \usepackage[normalem]{ulem} % ulem is needed to support strikethroughs (\sout)
                                % normalem makes italics be italics, not underlines
    \usepackage{mathrsfs}
    

    
    % Colors for the hyperref package
    \definecolor{urlcolor}{rgb}{0,.145,.698}
    \definecolor{linkcolor}{rgb}{.71,0.21,0.01}
    \definecolor{citecolor}{rgb}{.12,.54,.11}

    % ANSI colors
    \definecolor{ansi-black}{HTML}{3E424D}
    \definecolor{ansi-black-intense}{HTML}{282C36}
    \definecolor{ansi-red}{HTML}{E75C58}
    \definecolor{ansi-red-intense}{HTML}{B22B31}
    \definecolor{ansi-green}{HTML}{00A250}
    \definecolor{ansi-green-intense}{HTML}{007427}
    \definecolor{ansi-yellow}{HTML}{DDB62B}
    \definecolor{ansi-yellow-intense}{HTML}{B27D12}
    \definecolor{ansi-blue}{HTML}{208FFB}
    \definecolor{ansi-blue-intense}{HTML}{0065CA}
    \definecolor{ansi-magenta}{HTML}{D160C4}
    \definecolor{ansi-magenta-intense}{HTML}{A03196}
    \definecolor{ansi-cyan}{HTML}{60C6C8}
    \definecolor{ansi-cyan-intense}{HTML}{258F8F}
    \definecolor{ansi-white}{HTML}{C5C1B4}
    \definecolor{ansi-white-intense}{HTML}{A1A6B2}
    \definecolor{ansi-default-inverse-fg}{HTML}{FFFFFF}
    \definecolor{ansi-default-inverse-bg}{HTML}{000000}

    % commands and environments needed by pandoc snippets
    % extracted from the output of `pandoc -s`
    \providecommand{\tightlist}{%
      \setlength{\itemsep}{0pt}\setlength{\parskip}{0pt}}
    \DefineVerbatimEnvironment{Highlighting}{Verbatim}{commandchars=\\\{\}}
    % Add ',fontsize=\small' for more characters per line
    \newenvironment{Shaded}{}{}
    \newcommand{\KeywordTok}[1]{\textcolor[rgb]{0.00,0.44,0.13}{\textbf{{#1}}}}
    \newcommand{\DataTypeTok}[1]{\textcolor[rgb]{0.56,0.13,0.00}{{#1}}}
    \newcommand{\DecValTok}[1]{\textcolor[rgb]{0.25,0.63,0.44}{{#1}}}
    \newcommand{\BaseNTok}[1]{\textcolor[rgb]{0.25,0.63,0.44}{{#1}}}
    \newcommand{\FloatTok}[1]{\textcolor[rgb]{0.25,0.63,0.44}{{#1}}}
    \newcommand{\CharTok}[1]{\textcolor[rgb]{0.25,0.44,0.63}{{#1}}}
    \newcommand{\StringTok}[1]{\textcolor[rgb]{0.25,0.44,0.63}{{#1}}}
    \newcommand{\CommentTok}[1]{\textcolor[rgb]{0.38,0.63,0.69}{\textit{{#1}}}}
    \newcommand{\OtherTok}[1]{\textcolor[rgb]{0.00,0.44,0.13}{{#1}}}
    \newcommand{\AlertTok}[1]{\textcolor[rgb]{1.00,0.00,0.00}{\textbf{{#1}}}}
    \newcommand{\FunctionTok}[1]{\textcolor[rgb]{0.02,0.16,0.49}{{#1}}}
    \newcommand{\RegionMarkerTok}[1]{{#1}}
    \newcommand{\ErrorTok}[1]{\textcolor[rgb]{1.00,0.00,0.00}{\textbf{{#1}}}}
    \newcommand{\NormalTok}[1]{{#1}}
    
    % Additional commands for more recent versions of Pandoc
    \newcommand{\ConstantTok}[1]{\textcolor[rgb]{0.53,0.00,0.00}{{#1}}}
    \newcommand{\SpecialCharTok}[1]{\textcolor[rgb]{0.25,0.44,0.63}{{#1}}}
    \newcommand{\VerbatimStringTok}[1]{\textcolor[rgb]{0.25,0.44,0.63}{{#1}}}
    \newcommand{\SpecialStringTok}[1]{\textcolor[rgb]{0.73,0.40,0.53}{{#1}}}
    \newcommand{\ImportTok}[1]{{#1}}
    \newcommand{\DocumentationTok}[1]{\textcolor[rgb]{0.73,0.13,0.13}{\textit{{#1}}}}
    \newcommand{\AnnotationTok}[1]{\textcolor[rgb]{0.38,0.63,0.69}{\textbf{\textit{{#1}}}}}
    \newcommand{\CommentVarTok}[1]{\textcolor[rgb]{0.38,0.63,0.69}{\textbf{\textit{{#1}}}}}
    \newcommand{\VariableTok}[1]{\textcolor[rgb]{0.10,0.09,0.49}{{#1}}}
    \newcommand{\ControlFlowTok}[1]{\textcolor[rgb]{0.00,0.44,0.13}{\textbf{{#1}}}}
    \newcommand{\OperatorTok}[1]{\textcolor[rgb]{0.40,0.40,0.40}{{#1}}}
    \newcommand{\BuiltInTok}[1]{{#1}}
    \newcommand{\ExtensionTok}[1]{{#1}}
    \newcommand{\PreprocessorTok}[1]{\textcolor[rgb]{0.74,0.48,0.00}{{#1}}}
    \newcommand{\AttributeTok}[1]{\textcolor[rgb]{0.49,0.56,0.16}{{#1}}}
    \newcommand{\InformationTok}[1]{\textcolor[rgb]{0.38,0.63,0.69}{\textbf{\textit{{#1}}}}}
    \newcommand{\WarningTok}[1]{\textcolor[rgb]{0.38,0.63,0.69}{\textbf{\textit{{#1}}}}}
    
    
    % Define a nice break command that doesn't care if a line doesn't already
    % exist.
    \def\br{\hspace*{\fill} \\* }
    % Math Jax compatibility definitions
    \def\gt{>}
    \def\lt{<}
    \let\Oldtex\TeX
    \let\Oldlatex\LaTeX
    \renewcommand{\TeX}{\textrm{\Oldtex}}
    \renewcommand{\LaTeX}{\textrm{\Oldlatex}}
    % Document parameters
    % Document title
    \title{DecisionTree}
    
    
    
    
    
% Pygments definitions
\makeatletter
\def\PY@reset{\let\PY@it=\relax \let\PY@bf=\relax%
    \let\PY@ul=\relax \let\PY@tc=\relax%
    \let\PY@bc=\relax \let\PY@ff=\relax}
\def\PY@tok#1{\csname PY@tok@#1\endcsname}
\def\PY@toks#1+{\ifx\relax#1\empty\else%
    \PY@tok{#1}\expandafter\PY@toks\fi}
\def\PY@do#1{\PY@bc{\PY@tc{\PY@ul{%
    \PY@it{\PY@bf{\PY@ff{#1}}}}}}}
\def\PY#1#2{\PY@reset\PY@toks#1+\relax+\PY@do{#2}}

\expandafter\def\csname PY@tok@w\endcsname{\def\PY@tc##1{\textcolor[rgb]{0.73,0.73,0.73}{##1}}}
\expandafter\def\csname PY@tok@c\endcsname{\let\PY@it=\textit\def\PY@tc##1{\textcolor[rgb]{0.25,0.50,0.50}{##1}}}
\expandafter\def\csname PY@tok@cp\endcsname{\def\PY@tc##1{\textcolor[rgb]{0.74,0.48,0.00}{##1}}}
\expandafter\def\csname PY@tok@k\endcsname{\let\PY@bf=\textbf\def\PY@tc##1{\textcolor[rgb]{0.00,0.50,0.00}{##1}}}
\expandafter\def\csname PY@tok@kp\endcsname{\def\PY@tc##1{\textcolor[rgb]{0.00,0.50,0.00}{##1}}}
\expandafter\def\csname PY@tok@kt\endcsname{\def\PY@tc##1{\textcolor[rgb]{0.69,0.00,0.25}{##1}}}
\expandafter\def\csname PY@tok@o\endcsname{\def\PY@tc##1{\textcolor[rgb]{0.40,0.40,0.40}{##1}}}
\expandafter\def\csname PY@tok@ow\endcsname{\let\PY@bf=\textbf\def\PY@tc##1{\textcolor[rgb]{0.67,0.13,1.00}{##1}}}
\expandafter\def\csname PY@tok@nb\endcsname{\def\PY@tc##1{\textcolor[rgb]{0.00,0.50,0.00}{##1}}}
\expandafter\def\csname PY@tok@nf\endcsname{\def\PY@tc##1{\textcolor[rgb]{0.00,0.00,1.00}{##1}}}
\expandafter\def\csname PY@tok@nc\endcsname{\let\PY@bf=\textbf\def\PY@tc##1{\textcolor[rgb]{0.00,0.00,1.00}{##1}}}
\expandafter\def\csname PY@tok@nn\endcsname{\let\PY@bf=\textbf\def\PY@tc##1{\textcolor[rgb]{0.00,0.00,1.00}{##1}}}
\expandafter\def\csname PY@tok@ne\endcsname{\let\PY@bf=\textbf\def\PY@tc##1{\textcolor[rgb]{0.82,0.25,0.23}{##1}}}
\expandafter\def\csname PY@tok@nv\endcsname{\def\PY@tc##1{\textcolor[rgb]{0.10,0.09,0.49}{##1}}}
\expandafter\def\csname PY@tok@no\endcsname{\def\PY@tc##1{\textcolor[rgb]{0.53,0.00,0.00}{##1}}}
\expandafter\def\csname PY@tok@nl\endcsname{\def\PY@tc##1{\textcolor[rgb]{0.63,0.63,0.00}{##1}}}
\expandafter\def\csname PY@tok@ni\endcsname{\let\PY@bf=\textbf\def\PY@tc##1{\textcolor[rgb]{0.60,0.60,0.60}{##1}}}
\expandafter\def\csname PY@tok@na\endcsname{\def\PY@tc##1{\textcolor[rgb]{0.49,0.56,0.16}{##1}}}
\expandafter\def\csname PY@tok@nt\endcsname{\let\PY@bf=\textbf\def\PY@tc##1{\textcolor[rgb]{0.00,0.50,0.00}{##1}}}
\expandafter\def\csname PY@tok@nd\endcsname{\def\PY@tc##1{\textcolor[rgb]{0.67,0.13,1.00}{##1}}}
\expandafter\def\csname PY@tok@s\endcsname{\def\PY@tc##1{\textcolor[rgb]{0.73,0.13,0.13}{##1}}}
\expandafter\def\csname PY@tok@sd\endcsname{\let\PY@it=\textit\def\PY@tc##1{\textcolor[rgb]{0.73,0.13,0.13}{##1}}}
\expandafter\def\csname PY@tok@si\endcsname{\let\PY@bf=\textbf\def\PY@tc##1{\textcolor[rgb]{0.73,0.40,0.53}{##1}}}
\expandafter\def\csname PY@tok@se\endcsname{\let\PY@bf=\textbf\def\PY@tc##1{\textcolor[rgb]{0.73,0.40,0.13}{##1}}}
\expandafter\def\csname PY@tok@sr\endcsname{\def\PY@tc##1{\textcolor[rgb]{0.73,0.40,0.53}{##1}}}
\expandafter\def\csname PY@tok@ss\endcsname{\def\PY@tc##1{\textcolor[rgb]{0.10,0.09,0.49}{##1}}}
\expandafter\def\csname PY@tok@sx\endcsname{\def\PY@tc##1{\textcolor[rgb]{0.00,0.50,0.00}{##1}}}
\expandafter\def\csname PY@tok@m\endcsname{\def\PY@tc##1{\textcolor[rgb]{0.40,0.40,0.40}{##1}}}
\expandafter\def\csname PY@tok@gh\endcsname{\let\PY@bf=\textbf\def\PY@tc##1{\textcolor[rgb]{0.00,0.00,0.50}{##1}}}
\expandafter\def\csname PY@tok@gu\endcsname{\let\PY@bf=\textbf\def\PY@tc##1{\textcolor[rgb]{0.50,0.00,0.50}{##1}}}
\expandafter\def\csname PY@tok@gd\endcsname{\def\PY@tc##1{\textcolor[rgb]{0.63,0.00,0.00}{##1}}}
\expandafter\def\csname PY@tok@gi\endcsname{\def\PY@tc##1{\textcolor[rgb]{0.00,0.63,0.00}{##1}}}
\expandafter\def\csname PY@tok@gr\endcsname{\def\PY@tc##1{\textcolor[rgb]{1.00,0.00,0.00}{##1}}}
\expandafter\def\csname PY@tok@ge\endcsname{\let\PY@it=\textit}
\expandafter\def\csname PY@tok@gs\endcsname{\let\PY@bf=\textbf}
\expandafter\def\csname PY@tok@gp\endcsname{\let\PY@bf=\textbf\def\PY@tc##1{\textcolor[rgb]{0.00,0.00,0.50}{##1}}}
\expandafter\def\csname PY@tok@go\endcsname{\def\PY@tc##1{\textcolor[rgb]{0.53,0.53,0.53}{##1}}}
\expandafter\def\csname PY@tok@gt\endcsname{\def\PY@tc##1{\textcolor[rgb]{0.00,0.27,0.87}{##1}}}
\expandafter\def\csname PY@tok@err\endcsname{\def\PY@bc##1{\setlength{\fboxsep}{0pt}\fcolorbox[rgb]{1.00,0.00,0.00}{1,1,1}{\strut ##1}}}
\expandafter\def\csname PY@tok@kc\endcsname{\let\PY@bf=\textbf\def\PY@tc##1{\textcolor[rgb]{0.00,0.50,0.00}{##1}}}
\expandafter\def\csname PY@tok@kd\endcsname{\let\PY@bf=\textbf\def\PY@tc##1{\textcolor[rgb]{0.00,0.50,0.00}{##1}}}
\expandafter\def\csname PY@tok@kn\endcsname{\let\PY@bf=\textbf\def\PY@tc##1{\textcolor[rgb]{0.00,0.50,0.00}{##1}}}
\expandafter\def\csname PY@tok@kr\endcsname{\let\PY@bf=\textbf\def\PY@tc##1{\textcolor[rgb]{0.00,0.50,0.00}{##1}}}
\expandafter\def\csname PY@tok@bp\endcsname{\def\PY@tc##1{\textcolor[rgb]{0.00,0.50,0.00}{##1}}}
\expandafter\def\csname PY@tok@fm\endcsname{\def\PY@tc##1{\textcolor[rgb]{0.00,0.00,1.00}{##1}}}
\expandafter\def\csname PY@tok@vc\endcsname{\def\PY@tc##1{\textcolor[rgb]{0.10,0.09,0.49}{##1}}}
\expandafter\def\csname PY@tok@vg\endcsname{\def\PY@tc##1{\textcolor[rgb]{0.10,0.09,0.49}{##1}}}
\expandafter\def\csname PY@tok@vi\endcsname{\def\PY@tc##1{\textcolor[rgb]{0.10,0.09,0.49}{##1}}}
\expandafter\def\csname PY@tok@vm\endcsname{\def\PY@tc##1{\textcolor[rgb]{0.10,0.09,0.49}{##1}}}
\expandafter\def\csname PY@tok@sa\endcsname{\def\PY@tc##1{\textcolor[rgb]{0.73,0.13,0.13}{##1}}}
\expandafter\def\csname PY@tok@sb\endcsname{\def\PY@tc##1{\textcolor[rgb]{0.73,0.13,0.13}{##1}}}
\expandafter\def\csname PY@tok@sc\endcsname{\def\PY@tc##1{\textcolor[rgb]{0.73,0.13,0.13}{##1}}}
\expandafter\def\csname PY@tok@dl\endcsname{\def\PY@tc##1{\textcolor[rgb]{0.73,0.13,0.13}{##1}}}
\expandafter\def\csname PY@tok@s2\endcsname{\def\PY@tc##1{\textcolor[rgb]{0.73,0.13,0.13}{##1}}}
\expandafter\def\csname PY@tok@sh\endcsname{\def\PY@tc##1{\textcolor[rgb]{0.73,0.13,0.13}{##1}}}
\expandafter\def\csname PY@tok@s1\endcsname{\def\PY@tc##1{\textcolor[rgb]{0.73,0.13,0.13}{##1}}}
\expandafter\def\csname PY@tok@mb\endcsname{\def\PY@tc##1{\textcolor[rgb]{0.40,0.40,0.40}{##1}}}
\expandafter\def\csname PY@tok@mf\endcsname{\def\PY@tc##1{\textcolor[rgb]{0.40,0.40,0.40}{##1}}}
\expandafter\def\csname PY@tok@mh\endcsname{\def\PY@tc##1{\textcolor[rgb]{0.40,0.40,0.40}{##1}}}
\expandafter\def\csname PY@tok@mi\endcsname{\def\PY@tc##1{\textcolor[rgb]{0.40,0.40,0.40}{##1}}}
\expandafter\def\csname PY@tok@il\endcsname{\def\PY@tc##1{\textcolor[rgb]{0.40,0.40,0.40}{##1}}}
\expandafter\def\csname PY@tok@mo\endcsname{\def\PY@tc##1{\textcolor[rgb]{0.40,0.40,0.40}{##1}}}
\expandafter\def\csname PY@tok@ch\endcsname{\let\PY@it=\textit\def\PY@tc##1{\textcolor[rgb]{0.25,0.50,0.50}{##1}}}
\expandafter\def\csname PY@tok@cm\endcsname{\let\PY@it=\textit\def\PY@tc##1{\textcolor[rgb]{0.25,0.50,0.50}{##1}}}
\expandafter\def\csname PY@tok@cpf\endcsname{\let\PY@it=\textit\def\PY@tc##1{\textcolor[rgb]{0.25,0.50,0.50}{##1}}}
\expandafter\def\csname PY@tok@c1\endcsname{\let\PY@it=\textit\def\PY@tc##1{\textcolor[rgb]{0.25,0.50,0.50}{##1}}}
\expandafter\def\csname PY@tok@cs\endcsname{\let\PY@it=\textit\def\PY@tc##1{\textcolor[rgb]{0.25,0.50,0.50}{##1}}}

\def\PYZbs{\char`\\}
\def\PYZus{\char`\_}
\def\PYZob{\char`\{}
\def\PYZcb{\char`\}}
\def\PYZca{\char`\^}
\def\PYZam{\char`\&}
\def\PYZlt{\char`\<}
\def\PYZgt{\char`\>}
\def\PYZsh{\char`\#}
\def\PYZpc{\char`\%}
\def\PYZdl{\char`\$}
\def\PYZhy{\char`\-}
\def\PYZsq{\char`\'}
\def\PYZdq{\char`\"}
\def\PYZti{\char`\~}
% for compatibility with earlier versions
\def\PYZat{@}
\def\PYZlb{[}
\def\PYZrb{]}
\makeatother


    % For linebreaks inside Verbatim environment from package fancyvrb. 
    \makeatletter
        \newbox\Wrappedcontinuationbox 
        \newbox\Wrappedvisiblespacebox 
        \newcommand*\Wrappedvisiblespace {\textcolor{red}{\textvisiblespace}} 
        \newcommand*\Wrappedcontinuationsymbol {\textcolor{red}{\llap{\tiny$\m@th\hookrightarrow$}}} 
        \newcommand*\Wrappedcontinuationindent {3ex } 
        \newcommand*\Wrappedafterbreak {\kern\Wrappedcontinuationindent\copy\Wrappedcontinuationbox} 
        % Take advantage of the already applied Pygments mark-up to insert 
        % potential linebreaks for TeX processing. 
        %        {, <, #, %, $, ' and ": go to next line. 
        %        _, }, ^, &, >, - and ~: stay at end of broken line. 
        % Use of \textquotesingle for straight quote. 
        \newcommand*\Wrappedbreaksatspecials {% 
            \def\PYGZus{\discretionary{\char`\_}{\Wrappedafterbreak}{\char`\_}}% 
            \def\PYGZob{\discretionary{}{\Wrappedafterbreak\char`\{}{\char`\{}}% 
            \def\PYGZcb{\discretionary{\char`\}}{\Wrappedafterbreak}{\char`\}}}% 
            \def\PYGZca{\discretionary{\char`\^}{\Wrappedafterbreak}{\char`\^}}% 
            \def\PYGZam{\discretionary{\char`\&}{\Wrappedafterbreak}{\char`\&}}% 
            \def\PYGZlt{\discretionary{}{\Wrappedafterbreak\char`\<}{\char`\<}}% 
            \def\PYGZgt{\discretionary{\char`\>}{\Wrappedafterbreak}{\char`\>}}% 
            \def\PYGZsh{\discretionary{}{\Wrappedafterbreak\char`\#}{\char`\#}}% 
            \def\PYGZpc{\discretionary{}{\Wrappedafterbreak\char`\%}{\char`\%}}% 
            \def\PYGZdl{\discretionary{}{\Wrappedafterbreak\char`\$}{\char`\$}}% 
            \def\PYGZhy{\discretionary{\char`\-}{\Wrappedafterbreak}{\char`\-}}% 
            \def\PYGZsq{\discretionary{}{\Wrappedafterbreak\textquotesingle}{\textquotesingle}}% 
            \def\PYGZdq{\discretionary{}{\Wrappedafterbreak\char`\"}{\char`\"}}% 
            \def\PYGZti{\discretionary{\char`\~}{\Wrappedafterbreak}{\char`\~}}% 
        } 
        % Some characters . , ; ? ! / are not pygmentized. 
        % This macro makes them "active" and they will insert potential linebreaks 
        \newcommand*\Wrappedbreaksatpunct {% 
            \lccode`\~`\.\lowercase{\def~}{\discretionary{\hbox{\char`\.}}{\Wrappedafterbreak}{\hbox{\char`\.}}}% 
            \lccode`\~`\,\lowercase{\def~}{\discretionary{\hbox{\char`\,}}{\Wrappedafterbreak}{\hbox{\char`\,}}}% 
            \lccode`\~`\;\lowercase{\def~}{\discretionary{\hbox{\char`\;}}{\Wrappedafterbreak}{\hbox{\char`\;}}}% 
            \lccode`\~`\:\lowercase{\def~}{\discretionary{\hbox{\char`\:}}{\Wrappedafterbreak}{\hbox{\char`\:}}}% 
            \lccode`\~`\?\lowercase{\def~}{\discretionary{\hbox{\char`\?}}{\Wrappedafterbreak}{\hbox{\char`\?}}}% 
            \lccode`\~`\!\lowercase{\def~}{\discretionary{\hbox{\char`\!}}{\Wrappedafterbreak}{\hbox{\char`\!}}}% 
            \lccode`\~`\/\lowercase{\def~}{\discretionary{\hbox{\char`\/}}{\Wrappedafterbreak}{\hbox{\char`\/}}}% 
            \catcode`\.\active
            \catcode`\,\active 
            \catcode`\;\active
            \catcode`\:\active
            \catcode`\?\active
            \catcode`\!\active
            \catcode`\/\active 
            \lccode`\~`\~ 	
        }
    \makeatother

    \let\OriginalVerbatim=\Verbatim
    \makeatletter
    \renewcommand{\Verbatim}[1][1]{%
        %\parskip\z@skip
        \sbox\Wrappedcontinuationbox {\Wrappedcontinuationsymbol}%
        \sbox\Wrappedvisiblespacebox {\FV@SetupFont\Wrappedvisiblespace}%
        \def\FancyVerbFormatLine ##1{\hsize\linewidth
            \vtop{\raggedright\hyphenpenalty\z@\exhyphenpenalty\z@
                \doublehyphendemerits\z@\finalhyphendemerits\z@
                \strut ##1\strut}%
        }%
        % If the linebreak is at a space, the latter will be displayed as visible
        % space at end of first line, and a continuation symbol starts next line.
        % Stretch/shrink are however usually zero for typewriter font.
        \def\FV@Space {%
            \nobreak\hskip\z@ plus\fontdimen3\font minus\fontdimen4\font
            \discretionary{\copy\Wrappedvisiblespacebox}{\Wrappedafterbreak}
            {\kern\fontdimen2\font}%
        }%
        
        % Allow breaks at special characters using \PYG... macros.
        \Wrappedbreaksatspecials
        % Breaks at punctuation characters . , ; ? ! and / need catcode=\active 	
        \OriginalVerbatim[#1,codes*=\Wrappedbreaksatpunct]%
    }
    \makeatother

    % Exact colors from NB
    \definecolor{incolor}{HTML}{303F9F}
    \definecolor{outcolor}{HTML}{D84315}
    \definecolor{cellborder}{HTML}{CFCFCF}
    \definecolor{cellbackground}{HTML}{F7F7F7}
    
    % prompt
    \newcommand{\prompt}[4]{
        \llap{{\color{#2}[#3]: #4}}\vspace{-1.25em}
    }
    

    
    % Prevent overflowing lines due to hard-to-break entities
    \sloppy 
    % Setup hyperref package
    \hypersetup{
      breaklinks=true,  % so long urls are correctly broken across lines
      colorlinks=true,
      urlcolor=urlcolor,
      linkcolor=linkcolor,
      citecolor=citecolor,
      }
    % Slightly bigger margins than the latex defaults
    
    \geometry{verbose,tmargin=1in,bmargin=1in,lmargin=1in,rmargin=1in}
    
    

    \begin{document}
    
    
    % \maketitle
    \setcounter{page}{6}
    

    
    \hypertarget{ux51b3ux7b56ux6811ux7b97ux6cd5}{%
\section{决策树算法}\label{ux51b3ux7b56ux6811ux7b97ux6cd5}}

    \begin{tcolorbox}[breakable, size=fbox, boxrule=1pt, pad at break*=1mm,colback=cellbackground, colframe=cellborder]
\prompt{In}{incolor}{1}{\hspace{4pt}}
\begin{Verbatim}[commandchars=\\\{\}]
\PY{k+kn}{import} \PY{n+nn}{numpy} \PY{k}{as} \PY{n+nn}{np}
\PY{k+kn}{import} \PY{n+nn}{pandas} \PY{k}{as} \PY{n+nn}{pd}
\PY{k+kn}{import} \PY{n+nn}{json}
\end{Verbatim}
\end{tcolorbox}

    \hypertarget{ux6b63ux786eux5730ux8bfbux53d6ux6570ux636e}{%
\subsection{正确地读取数据}\label{ux6b63ux786eux5730ux8bfbux53d6ux6570ux636e}}

注意原始数据文件的格式,对其进行正确地处理后读入两个
DataFrame:\texttt{adult\_data\_df} 是训练集, \texttt{adult\_test\_df}
是测试集。DataFrame 中名为``50K''的列为标签(即分类)。

    \begin{tcolorbox}[breakable, size=fbox, boxrule=1pt, pad at break*=1mm,colback=cellbackground, colframe=cellborder]
\prompt{In}{incolor}{2}{\hspace{4pt}}
\begin{Verbatim}[commandchars=\\\{\}]
\PY{n}{col\PYZus{}names} \PY{o}{=} \PY{p}{[}\PY{l+s+s1}{\PYZsq{}}\PY{l+s+s1}{age}\PY{l+s+s1}{\PYZsq{}}\PY{p}{,} \PY{l+s+s1}{\PYZsq{}}\PY{l+s+s1}{workclass}\PY{l+s+s1}{\PYZsq{}}\PY{p}{,} \PY{l+s+s1}{\PYZsq{}}\PY{l+s+s1}{fnlwgt}\PY{l+s+s1}{\PYZsq{}}\PY{p}{,} \PY{l+s+s1}{\PYZsq{}}\PY{l+s+s1}{education}\PY{l+s+s1}{\PYZsq{}}\PY{p}{,} \PY{l+s+s1}{\PYZsq{}}\PY{l+s+s1}{education\PYZhy{}num}\PY{l+s+s1}{\PYZsq{}}\PY{p}{,} \PY{l+s+s1}{\PYZsq{}}\PY{l+s+s1}{marital\PYZhy{}status}\PY{l+s+s1}{\PYZsq{}}\PY{p}{,} \PY{l+s+s1}{\PYZsq{}}\PY{l+s+s1}{occupation}\PY{l+s+s1}{\PYZsq{}}\PY{p}{,} \PY{l+s+s1}{\PYZsq{}}\PY{l+s+s1}{relationship}\PY{l+s+s1}{\PYZsq{}}\PY{p}{,} \PY{l+s+s1}{\PYZsq{}}\PY{l+s+s1}{race}\PY{l+s+s1}{\PYZsq{}}\PY{p}{,} \PY{l+s+s1}{\PYZsq{}}\PY{l+s+s1}{sex}\PY{l+s+s1}{\PYZsq{}}\PY{p}{,} \PY{l+s+s1}{\PYZsq{}}\PY{l+s+s1}{capital\PYZhy{}gain}\PY{l+s+s1}{\PYZsq{}}\PY{p}{,} \PY{l+s+s1}{\PYZsq{}}\PY{l+s+s1}{capital\PYZhy{}loss}\PY{l+s+s1}{\PYZsq{}}\PY{p}{,} \PY{l+s+s1}{\PYZsq{}}\PY{l+s+s1}{hours\PYZhy{}per\PYZhy{}week}\PY{l+s+s1}{\PYZsq{}}\PY{p}{,} \PY{l+s+s1}{\PYZsq{}}\PY{l+s+s1}{native\PYZhy{}country}\PY{l+s+s1}{\PYZsq{}}\PY{p}{,} \PY{l+s+s1}{\PYZsq{}}\PY{l+s+s1}{50K}\PY{l+s+s1}{\PYZsq{}}\PY{p}{]}
\PY{n}{adult\PYZus{}data\PYZus{}df} \PY{o}{=} \PY{n}{pd}\PY{o}{.}\PY{n}{read\PYZus{}csv}\PY{p}{(}\PY{l+s+s1}{\PYZsq{}}\PY{l+s+s1}{dataset/adult.data}\PY{l+s+s1}{\PYZsq{}}\PY{p}{,} \PY{n}{index\PYZus{}col}\PY{o}{=}\PY{k+kc}{False}\PY{p}{,} \PY{n}{header}\PY{o}{=}\PY{k+kc}{None}\PY{p}{,} \PY{n}{names}\PY{o}{=}\PY{n}{col\PYZus{}names}\PY{p}{,} \PY{n}{sep}\PY{o}{=}\PY{l+s+s1}{\PYZsq{}}\PY{l+s+s1}{, }\PY{l+s+s1}{\PYZsq{}}\PY{p}{,} \PY{n}{engine}\PY{o}{=}\PY{l+s+s1}{\PYZsq{}}\PY{l+s+s1}{python}\PY{l+s+s1}{\PYZsq{}}\PY{p}{)}\PY{c+c1}{\PYZsh{}.drop([\PYZsq{}fnlwgt\PYZsq{}], axis=1)}
\PY{n}{adult\PYZus{}data\PYZus{}df}
\end{Verbatim}
\end{tcolorbox}

            \begin{tcolorbox}[breakable, boxrule=.5pt, size=fbox, pad at break*=1mm, opacityfill=0]
\prompt{Out}{outcolor}{2}{\hspace{3.5pt}}
\begin{Verbatim}[commandchars=\\\{\}]
       age         workclass  fnlwgt   education  education-num  \textbackslash{}
0       39         State-gov   77516   Bachelors             13
1       50  Self-emp-not-inc   83311   Bachelors             13
2       38           Private  215646     HS-grad              9
3       53           Private  234721        11th              7
4       28           Private  338409   Bachelors             13
{\ldots}    {\ldots}               {\ldots}     {\ldots}         {\ldots}            {\ldots}
32556   27           Private  257302  Assoc-acdm             12
32557   40           Private  154374     HS-grad              9
32558   58           Private  151910     HS-grad              9
32559   22           Private  201490     HS-grad              9
32560   52      Self-emp-inc  287927     HS-grad              9

           marital-status         occupation   relationship   race     sex  \textbackslash{}
0           Never-married       Adm-clerical  Not-in-family  White    Male
1      Married-civ-spouse    Exec-managerial        Husband  White    Male
2                Divorced  Handlers-cleaners  Not-in-family  White    Male
3      Married-civ-spouse  Handlers-cleaners        Husband  Black    Male
4      Married-civ-spouse     Prof-specialty           Wife  Black  Female
{\ldots}                   {\ldots}                {\ldots}            {\ldots}    {\ldots}     {\ldots}
32556  Married-civ-spouse       Tech-support           Wife  White  Female
32557  Married-civ-spouse  Machine-op-inspct        Husband  White    Male
32558             Widowed       Adm-clerical      Unmarried  White  Female
32559       Never-married       Adm-clerical      Own-child  White    Male
32560  Married-civ-spouse    Exec-managerial           Wife  White  Female

       capital-gain  capital-loss  hours-per-week native-country    50K
0              2174             0              40  United-States  <=50K
1                 0             0              13  United-States  <=50K
2                 0             0              40  United-States  <=50K
3                 0             0              40  United-States  <=50K
4                 0             0              40           Cuba  <=50K
{\ldots}             {\ldots}           {\ldots}             {\ldots}            {\ldots}    {\ldots}
32556             0             0              38  United-States  <=50K
32557             0             0              40  United-States   >50K
32558             0             0              40  United-States  <=50K
32559             0             0              20  United-States  <=50K
32560         15024             0              40  United-States   >50K

[32561 rows x 15 columns]
\end{Verbatim}
\end{tcolorbox}
        
    \begin{tcolorbox}[breakable, size=fbox, boxrule=1pt, pad at break*=1mm,colback=cellbackground, colframe=cellborder]
\prompt{In}{incolor}{3}{\hspace{4pt}}
\begin{Verbatim}[commandchars=\\\{\}]
\PY{n}{adult\PYZus{}test\PYZus{}df} \PY{o}{=} \PY{n}{pd}\PY{o}{.}\PY{n}{read\PYZus{}csv}\PY{p}{(}\PY{l+s+s1}{\PYZsq{}}\PY{l+s+s1}{dataset/adult.test}\PY{l+s+s1}{\PYZsq{}}\PY{p}{,} \PY{n}{skiprows}\PY{o}{=}\PY{p}{[}\PY{l+m+mi}{0}\PY{p}{]}\PY{p}{,} \PY{n}{index\PYZus{}col}\PY{o}{=}\PY{k+kc}{False}\PY{p}{,} \PY{n}{header}\PY{o}{=}\PY{k+kc}{None}\PY{p}{,} \PY{n}{names}\PY{o}{=}\PY{n}{col\PYZus{}names}\PY{p}{,} \PY{n}{sep}\PY{o}{=}\PY{l+s+s1}{\PYZsq{}}\PY{l+s+s1}{, }\PY{l+s+s1}{\PYZsq{}}\PY{p}{,} \PY{n}{engine}\PY{o}{=}\PY{l+s+s1}{\PYZsq{}}\PY{l+s+s1}{python}\PY{l+s+s1}{\PYZsq{}}\PY{p}{)}\PY{c+c1}{\PYZsh{}.drop([\PYZsq{}fnlwgt\PYZsq{}], axis=1)}
\PY{n}{adult\PYZus{}test\PYZus{}df}\PY{p}{[}\PY{l+s+s1}{\PYZsq{}}\PY{l+s+s1}{50K}\PY{l+s+s1}{\PYZsq{}}\PY{p}{]} \PY{o}{=} \PY{n}{adult\PYZus{}test\PYZus{}df}\PY{p}{[}\PY{l+s+s1}{\PYZsq{}}\PY{l+s+s1}{50K}\PY{l+s+s1}{\PYZsq{}}\PY{p}{]}\PY{o}{.}\PY{n}{map}\PY{p}{(}\PY{k}{lambda} \PY{n}{x}\PY{p}{:} \PY{n}{x}\PY{p}{[}\PY{p}{:}\PY{o}{\PYZhy{}}\PY{l+m+mi}{1}\PY{p}{]}\PY{p}{)}  \PY{c+c1}{\PYZsh{} 去除行末的点}
\PY{n}{adult\PYZus{}test\PYZus{}df}
\end{Verbatim}
\end{tcolorbox}

            \begin{tcolorbox}[breakable, boxrule=.5pt, size=fbox, pad at break*=1mm, opacityfill=0]
\prompt{Out}{outcolor}{3}{\hspace{3.5pt}}
\begin{Verbatim}[commandchars=\\\{\}]
       age     workclass  fnlwgt     education  education-num  \textbackslash{}
0       25       Private  226802          11th              7
1       38       Private   89814       HS-grad              9
2       28     Local-gov  336951    Assoc-acdm             12
3       44       Private  160323  Some-college             10
4       18             ?  103497  Some-college             10
{\ldots}    {\ldots}           {\ldots}     {\ldots}           {\ldots}            {\ldots}
16276   39       Private  215419     Bachelors             13
16277   64             ?  321403       HS-grad              9
16278   38       Private  374983     Bachelors             13
16279   44       Private   83891     Bachelors             13
16280   35  Self-emp-inc  182148     Bachelors             13

           marital-status         occupation    relationship  \textbackslash{}
0           Never-married  Machine-op-inspct       Own-child
1      Married-civ-spouse    Farming-fishing         Husband
2      Married-civ-spouse    Protective-serv         Husband
3      Married-civ-spouse  Machine-op-inspct         Husband
4           Never-married                  ?       Own-child
{\ldots}                   {\ldots}                {\ldots}             {\ldots}
16276            Divorced     Prof-specialty   Not-in-family
16277             Widowed                  ?  Other-relative
16278  Married-civ-spouse     Prof-specialty         Husband
16279            Divorced       Adm-clerical       Own-child
16280  Married-civ-spouse    Exec-managerial         Husband

                     race     sex  capital-gain  capital-loss  hours-per-week  \textbackslash{}
0                   Black    Male             0             0              40
1                   White    Male             0             0              50
2                   White    Male             0             0              40
3                   Black    Male          7688             0              40
4                   White  Female             0             0              30
{\ldots}                   {\ldots}     {\ldots}           {\ldots}           {\ldots}             {\ldots}
16276               White  Female             0             0              36
16277               Black    Male             0             0              40
16278               White    Male             0             0              50
16279  Asian-Pac-Islander    Male          5455             0              40
16280               White    Male             0             0              60

      native-country    50K
0      United-States  <=50K
1      United-States  <=50K
2      United-States   >50K
3      United-States   >50K
4      United-States  <=50K
{\ldots}              {\ldots}    {\ldots}
16276  United-States  <=50K
16277  United-States  <=50K
16278  United-States  <=50K
16279  United-States  <=50K
16280  United-States   >50K

[16281 rows x 15 columns]
\end{Verbatim}
\end{tcolorbox}
        
    \hypertarget{ux8865ux5145ux7f3aux5931ux503c}{%
\subsection{补充缺失值}\label{ux8865ux5145ux7f3aux5931ux503c}}

通过对数据的基本观察得知,缺失值所在的列均为离散属性,因此只需要对离散缺失值进行补全即可,本例数据集上无需考虑连续型数据的补全。我采用的方法是使用该列出现次数最多的值(即众数)代替缺失值。

    \begin{tcolorbox}[breakable, size=fbox, boxrule=1pt, pad at break*=1mm,colback=cellbackground, colframe=cellborder]
\prompt{In}{incolor}{4}{\hspace{4pt}}
\begin{Verbatim}[commandchars=\\\{\}]
\PY{c+c1}{\PYZsh{} 补充缺失值,}
\PY{n+nb}{print}\PY{p}{(}\PY{l+s+s1}{\PYZsq{}}\PY{l+s+s1}{[adult.data]}\PY{l+s+s1}{\PYZsq{}}\PY{p}{)}
\PY{n}{mode\PYZus{}df} \PY{o}{=} \PY{n}{adult\PYZus{}data\PYZus{}df}\PY{o}{.}\PY{n}{mode}\PY{p}{(}\PY{p}{)}  \PY{c+c1}{\PYZsh{} 众数}
\PY{k}{for} \PY{n}{col} \PY{o+ow}{in} \PY{n}{adult\PYZus{}data\PYZus{}df}\PY{p}{:}
    \PY{k}{if} \PY{l+s+s1}{\PYZsq{}}\PY{l+s+s1}{?}\PY{l+s+s1}{\PYZsq{}} \PY{o+ow}{in} \PY{n}{adult\PYZus{}data\PYZus{}df}\PY{p}{[}\PY{n}{col}\PY{p}{]}\PY{o}{.}\PY{n}{tolist}\PY{p}{(}\PY{p}{)}\PY{p}{:}
        \PY{n}{missing\PYZus{}count} \PY{o}{=} \PY{n}{adult\PYZus{}data\PYZus{}df}\PY{p}{[}\PY{n}{col}\PY{p}{]}\PY{o}{.}\PY{n}{value\PYZus{}counts}\PY{p}{(}\PY{p}{)}\PY{p}{[}\PY{l+s+s1}{\PYZsq{}}\PY{l+s+s1}{?}\PY{l+s+s1}{\PYZsq{}}\PY{p}{]}   \PY{c+c1}{\PYZsh{} 缺失值的个数}
        \PY{n}{adult\PYZus{}data\PYZus{}df}\PY{p}{[}\PY{n}{col}\PY{p}{]} \PY{o}{=} \PY{n}{adult\PYZus{}data\PYZus{}df}\PY{p}{[}\PY{n}{col}\PY{p}{]}\PY{o}{.}\PY{n}{replace}\PY{p}{(}\PY{l+s+s1}{\PYZsq{}}\PY{l+s+s1}{?}\PY{l+s+s1}{\PYZsq{}}\PY{p}{,} \PY{n}{mode\PYZus{}df}\PY{p}{[}\PY{n}{col}\PY{p}{]}\PY{p}{[}\PY{l+m+mi}{0}\PY{p}{]}\PY{p}{)}
        \PY{n+nb}{print}\PY{p}{(}\PY{l+s+s1}{\PYZsq{}}\PY{l+s+si}{\PYZob{}\PYZcb{}}\PY{l+s+s1}{: }\PY{l+s+si}{\PYZob{}\PYZcb{}}\PY{l+s+s1}{ missing values are replaced with }\PY{l+s+s1}{\PYZdq{}}\PY{l+s+si}{\PYZob{}\PYZcb{}}\PY{l+s+s1}{\PYZdq{}}\PY{l+s+s1}{\PYZsq{}}\PY{o}{.}\PY{n}{format}\PY{p}{(}\PY{n}{col}\PY{p}{,} \PY{n}{missing\PYZus{}count}\PY{p}{,} \PY{n}{mode\PYZus{}df}\PY{p}{[}\PY{n}{col}\PY{p}{]}\PY{p}{[}\PY{l+m+mi}{0}\PY{p}{]}\PY{p}{)}\PY{p}{)}

\PY{n+nb}{print}\PY{p}{(}\PY{l+s+s1}{\PYZsq{}}\PY{l+s+s1}{\PYZhy{}\PYZhy{}\PYZhy{}\PYZhy{}\PYZhy{}\PYZhy{}\PYZhy{}\PYZhy{}\PYZhy{}\PYZhy{}\PYZhy{}\PYZhy{}\PYZhy{}\PYZhy{}\PYZhy{}\PYZhy{}\PYZhy{}\PYZhy{}\PYZhy{}\PYZhy{}\PYZhy{}\PYZhy{}\PYZhy{}\PYZhy{}\PYZhy{}\PYZhy{}\PYZhy{}\PYZhy{}\PYZhy{}\PYZhy{}\PYZhy{}}\PY{l+s+s1}{\PYZsq{}}\PY{p}{)}
\PY{n+nb}{print}\PY{p}{(}\PY{l+s+s1}{\PYZsq{}}\PY{l+s+s1}{[adult.test]}\PY{l+s+s1}{\PYZsq{}}\PY{p}{)}
\PY{n}{mode\PYZus{}df} \PY{o}{=} \PY{n}{adult\PYZus{}test\PYZus{}df}\PY{o}{.}\PY{n}{mode}\PY{p}{(}\PY{p}{)}  \PY{c+c1}{\PYZsh{} 众数}
\PY{k}{for} \PY{n}{col} \PY{o+ow}{in} \PY{n}{adult\PYZus{}test\PYZus{}df}\PY{p}{:}
    \PY{k}{if} \PY{l+s+s1}{\PYZsq{}}\PY{l+s+s1}{?}\PY{l+s+s1}{\PYZsq{}} \PY{o+ow}{in} \PY{n}{adult\PYZus{}test\PYZus{}df}\PY{p}{[}\PY{n}{col}\PY{p}{]}\PY{o}{.}\PY{n}{tolist}\PY{p}{(}\PY{p}{)}\PY{p}{:}
        \PY{n}{missing\PYZus{}count} \PY{o}{=} \PY{n}{adult\PYZus{}test\PYZus{}df}\PY{p}{[}\PY{n}{col}\PY{p}{]}\PY{o}{.}\PY{n}{value\PYZus{}counts}\PY{p}{(}\PY{p}{)}\PY{p}{[}\PY{l+s+s1}{\PYZsq{}}\PY{l+s+s1}{?}\PY{l+s+s1}{\PYZsq{}}\PY{p}{]}   \PY{c+c1}{\PYZsh{} 缺失值的个数}
        \PY{n}{adult\PYZus{}test\PYZus{}df}\PY{p}{[}\PY{n}{col}\PY{p}{]} \PY{o}{=} \PY{n}{adult\PYZus{}test\PYZus{}df}\PY{p}{[}\PY{n}{col}\PY{p}{]}\PY{o}{.}\PY{n}{replace}\PY{p}{(}\PY{l+s+s1}{\PYZsq{}}\PY{l+s+s1}{?}\PY{l+s+s1}{\PYZsq{}}\PY{p}{,} \PY{n}{mode\PYZus{}df}\PY{p}{[}\PY{n}{col}\PY{p}{]}\PY{p}{[}\PY{l+m+mi}{0}\PY{p}{]}\PY{p}{)}
        \PY{n+nb}{print}\PY{p}{(}\PY{l+s+s1}{\PYZsq{}}\PY{l+s+si}{\PYZob{}\PYZcb{}}\PY{l+s+s1}{: }\PY{l+s+si}{\PYZob{}\PYZcb{}}\PY{l+s+s1}{ missing values are replaced with }\PY{l+s+s1}{\PYZdq{}}\PY{l+s+si}{\PYZob{}\PYZcb{}}\PY{l+s+s1}{\PYZdq{}}\PY{l+s+s1}{\PYZsq{}}\PY{o}{.}\PY{n}{format}\PY{p}{(}\PY{n}{col}\PY{p}{,} \PY{n}{missing\PYZus{}count}\PY{p}{,} \PY{n}{mode\PYZus{}df}\PY{p}{[}\PY{n}{col}\PY{p}{]}\PY{p}{[}\PY{l+m+mi}{0}\PY{p}{]}\PY{p}{)}\PY{p}{)}
\end{Verbatim}
\end{tcolorbox}

    \begin{Verbatim}[commandchars=\\\{\}]
[adult.data]
workclass: 1836 missing values are replaced with "Private"
occupation: 1843 missing values are replaced with "Prof-specialty"
native-country: 583 missing values are replaced with "United-States"
-------------------------------
[adult.test]
workclass: 963 missing values are replaced with "Private"
occupation: 966 missing values are replaced with "Prof-specialty"
native-country: 274 missing values are replaced with "United-States"
\end{Verbatim}

    \hypertarget{ux5904ux7406ux8fdeux7eedux578bux53d8ux91cf}{%
\subsection{处理连续型变量}\label{ux5904ux7406ux8fdeux7eedux578bux53d8ux91cf}}

需要将连续型变量离散化,离散化方法是二分法(bi-partition),选取使得划分后信息增益最大的点作为划分点。方法详见``西瓜书''第
4.4 节。

    \begin{tcolorbox}[breakable, size=fbox, boxrule=1pt, pad at break*=1mm,colback=cellbackground, colframe=cellborder]
\prompt{In}{incolor}{5}{\hspace{4pt}}
\begin{Verbatim}[commandchars=\\\{\}]
\PY{k}{def} \PY{n+nf}{entropy}\PY{p}{(}\PY{n}{df}\PY{p}{)}\PY{p}{:}
    \PY{l+s+sd}{\PYZdq{}\PYZdq{}\PYZdq{}计算信息熵。}
\PY{l+s+sd}{    Args:}
\PY{l+s+sd}{        df: 要计算信息熵的二分类数据集。}
\PY{l+s+sd}{    Returns:}
\PY{l+s+sd}{        信息熵值。}
\PY{l+s+sd}{    \PYZdq{}\PYZdq{}\PYZdq{}}
    \PY{k}{try}\PY{p}{:}
        \PY{n}{q} \PY{o}{=} \PY{n}{df}\PY{p}{[}\PY{l+s+s1}{\PYZsq{}}\PY{l+s+s1}{50K}\PY{l+s+s1}{\PYZsq{}}\PY{p}{]}\PY{o}{.}\PY{n}{value\PYZus{}counts}\PY{p}{(}\PY{p}{)}\PY{p}{[}\PY{l+s+s1}{\PYZsq{}}\PY{l+s+s1}{\PYZlt{}=50K}\PY{l+s+s1}{\PYZsq{}}\PY{p}{]} \PY{o}{/} \PY{n+nb}{len}\PY{p}{(}\PY{n}{df}\PY{p}{[}\PY{l+s+s1}{\PYZsq{}}\PY{l+s+s1}{50K}\PY{l+s+s1}{\PYZsq{}}\PY{p}{]}\PY{p}{)}  \PY{c+c1}{\PYZsh{} 正样本的概率}
    \PY{k}{except}\PY{p}{:}
        \PY{n}{q} \PY{o}{=} \PY{l+m+mi}{0}
    \PY{k}{if} \PY{n}{q} \PY{o}{==} \PY{l+m+mi}{0} \PY{o+ow}{or} \PY{n}{q} \PY{o}{==} \PY{l+m+mi}{1}\PY{p}{:}
        \PY{k}{return} \PY{l+m+mi}{0}  \PY{c+c1}{\PYZsh{} 约定}
    \PY{k}{else}\PY{p}{:}
        \PY{k}{return} \PY{o}{\PYZhy{}}\PY{p}{(}\PY{n}{q} \PY{o}{*} \PY{n}{np}\PY{o}{.}\PY{n}{log2}\PY{p}{(}\PY{n}{q}\PY{p}{)} \PY{o}{+} \PY{p}{(}\PY{l+m+mi}{1}\PY{o}{\PYZhy{}}\PY{n}{q}\PY{p}{)} \PY{o}{*} \PY{n}{np}\PY{o}{.}\PY{n}{log2}\PY{p}{(}\PY{l+m+mi}{1}\PY{o}{\PYZhy{}}\PY{n}{q}\PY{p}{)}\PY{p}{)}

\PY{k}{def} \PY{n+nf}{informationGain}\PY{p}{(}\PY{n}{df}\PY{p}{,} \PY{n}{attribute}\PY{p}{)}\PY{p}{:}
    \PY{l+s+sd}{\PYZdq{}\PYZdq{}\PYZdq{}计算信息增益。}
\PY{l+s+sd}{    Args:}
\PY{l+s+sd}{        df: 数据集。}
\PY{l+s+sd}{        attribute: 选取的属性。}
\PY{l+s+sd}{    Returns:}
\PY{l+s+sd}{        信息增益值。}
\PY{l+s+sd}{    \PYZdq{}\PYZdq{}\PYZdq{}}
    \PY{n}{remainder} \PY{o}{=} \PY{l+m+mi}{0}  \PY{c+c1}{\PYZsh{} 累积条件熵}
    \PY{c+c1}{\PYZsh{} 对指定属性的每个取值value}
    \PY{k}{for} \PY{n}{value} \PY{o+ow}{in} \PY{n}{df}\PY{p}{[}\PY{n}{attribute}\PY{p}{]}\PY{o}{.}\PY{n}{unique}\PY{p}{(}\PY{p}{)}\PY{p}{:}
        \PY{n}{sub\PYZus{}df} \PY{o}{=} \PY{n}{df}\PY{p}{[}\PY{n}{df}\PY{p}{[}\PY{n}{attribute}\PY{p}{]}\PY{o}{==}\PY{n}{value}\PY{p}{]}
        \PY{n}{remainder} \PY{o}{+}\PY{o}{=} \PY{n+nb}{len}\PY{p}{(}\PY{n}{sub\PYZus{}df}\PY{p}{)}\PY{o}{/}\PY{n+nb}{len}\PY{p}{(}\PY{n}{df}\PY{p}{)} \PY{o}{*} \PY{n}{entropy}\PY{p}{(}\PY{n}{sub\PYZus{}df}\PY{p}{)}
    \PY{k}{return} \PY{n}{entropy}\PY{p}{(}\PY{n}{df}\PY{p}{)} \PY{o}{\PYZhy{}} \PY{n}{remainder}  \PY{c+c1}{\PYZsh{} 信息熵 \PYZhy{} 条件熵}
\end{Verbatim}
\end{tcolorbox}

    \begin{tcolorbox}[breakable, size=fbox, boxrule=1pt, pad at break*=1mm,colback=cellbackground, colframe=cellborder]
\prompt{In}{incolor}{6}{\hspace{4pt}}
\begin{Verbatim}[commandchars=\\\{\}]
\PY{n}{continuous\PYZus{}attrs} \PY{o}{=} \PY{p}{[}\PY{l+s+s1}{\PYZsq{}}\PY{l+s+s1}{age}\PY{l+s+s1}{\PYZsq{}}\PY{p}{,} \PY{l+s+s1}{\PYZsq{}}\PY{l+s+s1}{fnlwgt}\PY{l+s+s1}{\PYZsq{}}\PY{p}{,} \PY{l+s+s1}{\PYZsq{}}\PY{l+s+s1}{education\PYZhy{}num}\PY{l+s+s1}{\PYZsq{}}\PY{p}{,} \PY{l+s+s1}{\PYZsq{}}\PY{l+s+s1}{capital\PYZhy{}gain}\PY{l+s+s1}{\PYZsq{}}\PY{p}{,} \PY{l+s+s1}{\PYZsq{}}\PY{l+s+s1}{capital\PYZhy{}loss}\PY{l+s+s1}{\PYZsq{}}\PY{p}{,} \PY{l+s+s1}{\PYZsq{}}\PY{l+s+s1}{hours\PYZhy{}per\PYZhy{}week}\PY{l+s+s1}{\PYZsq{}}\PY{p}{]}  \PY{c+c1}{\PYZsh{} 连续型属性}

\PY{k}{for} \PY{n}{attr} \PY{o+ow}{in} \PY{n}{continuous\PYZus{}attrs}\PY{p}{:}
    \PY{n}{partition\PYZus{}point} \PY{o}{=} \PY{o}{\PYZhy{}}\PY{l+m+mi}{1}
    \PY{n}{max\PYZus{}ig} \PY{o}{=} \PY{l+m+mi}{0}

    \PY{c+c1}{\PYZsh{} 在训练集上尝试以每个值进行划分,选出信息增益最大的那个划分点}
    \PY{k}{for} \PY{n}{value} \PY{o+ow}{in} \PY{n+nb}{sorted}\PY{p}{(}\PY{n+nb}{list}\PY{p}{(}\PY{n}{adult\PYZus{}data\PYZus{}df}\PY{p}{[}\PY{n}{attr}\PY{p}{]}\PY{o}{.}\PY{n}{unique}\PY{p}{(}\PY{p}{)}\PY{p}{)}\PY{p}{)}\PY{p}{:}
        \PY{n}{adult\PYZus{}data\PYZus{}df}\PY{p}{[}\PY{l+s+s1}{\PYZsq{}}\PY{l+s+s1}{temp}\PY{l+s+s1}{\PYZsq{}}\PY{p}{]} \PY{o}{=} \PY{n}{adult\PYZus{}data\PYZus{}df}\PY{p}{[}\PY{n}{attr}\PY{p}{]}\PY{o}{.}\PY{n}{map}\PY{p}{(}\PY{k}{lambda} \PY{n}{x}\PY{p}{:} \PY{l+s+s1}{\PYZsq{}}\PY{l+s+s1}{+}\PY{l+s+s1}{\PYZsq{}} \PY{k}{if} \PY{n}{x}\PY{o}{\PYZgt{}}\PY{n}{value} \PY{k}{else} \PY{l+s+s1}{\PYZsq{}}\PY{l+s+s1}{\PYZhy{}}\PY{l+s+s1}{\PYZsq{}}\PY{p}{)}  \PY{c+c1}{\PYZsh{} 大于划分点表示为\PYZsq{}+\PYZsq{},小于等于划分点表示为\PYZsq{}\PYZhy{}\PYZsq{}}
        \PY{n}{current\PYZus{}ig} \PY{o}{=} \PY{n}{informationGain}\PY{p}{(}\PY{n}{adult\PYZus{}data\PYZus{}df}\PY{p}{,} \PY{l+s+s1}{\PYZsq{}}\PY{l+s+s1}{temp}\PY{l+s+s1}{\PYZsq{}}\PY{p}{)}  \PY{c+c1}{\PYZsh{} 计算当前划分的信息增益}
        \PY{k}{if} \PY{n}{current\PYZus{}ig} \PY{o}{\PYZgt{}}\PY{o}{=} \PY{n}{max\PYZus{}ig}\PY{p}{:}
            \PY{n}{partition\PYZus{}point} \PY{o}{=} \PY{n}{value}
            \PY{n}{max\PYZus{}ig} \PY{o}{=} \PY{n}{current\PYZus{}ig}
    \PY{n}{adult\PYZus{}data\PYZus{}df}\PY{o}{.}\PY{n}{drop}\PY{p}{(}\PY{p}{[}\PY{l+s+s1}{\PYZsq{}}\PY{l+s+s1}{temp}\PY{l+s+s1}{\PYZsq{}}\PY{p}{]}\PY{p}{,} \PY{n}{axis}\PY{o}{=}\PY{l+m+mi}{1}\PY{p}{,} \PY{n}{inplace}\PY{o}{=}\PY{k+kc}{True}\PY{p}{)}  \PY{c+c1}{\PYZsh{} 删掉临时属性列}

    \PY{c+c1}{\PYZsh{} 用同样的划分点离散化训练集和测试集}
    \PY{n}{adult\PYZus{}data\PYZus{}df}\PY{p}{[}\PY{n}{attr}\PY{p}{]} \PY{o}{=} \PY{n}{adult\PYZus{}data\PYZus{}df}\PY{p}{[}\PY{n}{attr}\PY{p}{]}\PY{o}{.}\PY{n}{map}\PY{p}{(}\PY{k}{lambda} \PY{n}{x}\PY{p}{:} \PY{l+s+s1}{\PYZsq{}}\PY{l+s+si}{\PYZob{}\PYZcb{}}\PY{l+s+s1}{+}\PY{l+s+s1}{\PYZsq{}}\PY{o}{.}\PY{n}{format}\PY{p}{(}\PY{n}{partition\PYZus{}point}\PY{p}{)} \PY{k}{if} \PY{n}{x}\PY{o}{\PYZgt{}}\PY{n}{partition\PYZus{}point} \PY{k}{else} \PY{l+s+s1}{\PYZsq{}}\PY{l+s+si}{\PYZob{}\PYZcb{}}\PY{l+s+s1}{\PYZhy{}}\PY{l+s+s1}{\PYZsq{}}\PY{o}{.}\PY{n}{format}\PY{p}{(}\PY{n}{partition\PYZus{}point}\PY{p}{)}\PY{p}{)}
    \PY{n}{adult\PYZus{}test\PYZus{}df}\PY{p}{[}\PY{n}{attr}\PY{p}{]} \PY{o}{=} \PY{n}{adult\PYZus{}test\PYZus{}df}\PY{p}{[}\PY{n}{attr}\PY{p}{]}\PY{o}{.}\PY{n}{map}\PY{p}{(}\PY{k}{lambda} \PY{n}{x}\PY{p}{:} \PY{l+s+s1}{\PYZsq{}}\PY{l+s+si}{\PYZob{}\PYZcb{}}\PY{l+s+s1}{+}\PY{l+s+s1}{\PYZsq{}}\PY{o}{.}\PY{n}{format}\PY{p}{(}\PY{n}{partition\PYZus{}point}\PY{p}{)} \PY{k}{if} \PY{n}{x}\PY{o}{\PYZgt{}}\PY{n}{partition\PYZus{}point} \PY{k}{else} \PY{l+s+s1}{\PYZsq{}}\PY{l+s+si}{\PYZob{}\PYZcb{}}\PY{l+s+s1}{\PYZhy{}}\PY{l+s+s1}{\PYZsq{}}\PY{o}{.}\PY{n}{format}\PY{p}{(}\PY{n}{partition\PYZus{}point}\PY{p}{)}\PY{p}{)}
    \PY{n+nb}{print}\PY{p}{(}\PY{n}{attr}\PY{p}{,} \PY{n}{partition\PYZus{}point}\PY{p}{)}  \PY{c+c1}{\PYZsh{} debug}

\PY{c+c1}{\PYZsh{} 保存离散化后的数据集,方便下次使用}
\PY{n}{adult\PYZus{}data\PYZus{}df}\PY{o}{.}\PY{n}{to\PYZus{}csv}\PY{p}{(}\PY{l+s+s1}{\PYZsq{}}\PY{l+s+s1}{dataset/discretized\PYZus{}adult.data}\PY{l+s+s1}{\PYZsq{}}\PY{p}{,} \PY{n}{index}\PY{o}{=}\PY{k+kc}{False}\PY{p}{)}
\PY{n}{adult\PYZus{}test\PYZus{}df}\PY{o}{.}\PY{n}{to\PYZus{}csv}\PY{p}{(}\PY{l+s+s1}{\PYZsq{}}\PY{l+s+s1}{dataset/discretized\PYZus{}adult.test}\PY{l+s+s1}{\PYZsq{}}\PY{p}{,} \PY{n}{index}\PY{o}{=}\PY{k+kc}{False}\PY{p}{)}
\end{Verbatim}
\end{tcolorbox}

    \begin{Verbatim}[commandchars=\\\{\}]
age 27
fnlwgt 209912
education-num 12
capital-gain 6849
capital-loss 1816
hours-per-week 41
\end{Verbatim}

    上面步骤中,对 \texttt{fnlwgt}
属性的处理很慢。然而实验结果表明,即使不考虑该属性,对模型准确性也不会产生明显影响。

    \begin{tcolorbox}[breakable, size=fbox, boxrule=1pt, pad at break*=1mm,colback=cellbackground, colframe=cellborder]
\prompt{In}{incolor}{7}{\hspace{4pt}}
\begin{Verbatim}[commandchars=\\\{\}]
\PY{c+c1}{\PYZsh{} 从文件中读取预处理过的数据集}
\PY{n}{adult\PYZus{}data\PYZus{}df} \PY{o}{=} \PY{n}{pd}\PY{o}{.}\PY{n}{read\PYZus{}csv}\PY{p}{(}\PY{l+s+s1}{\PYZsq{}}\PY{l+s+s1}{dataset/discretized\PYZus{}adult.data}\PY{l+s+s1}{\PYZsq{}}\PY{p}{)}
\PY{n}{adult\PYZus{}test\PYZus{}df} \PY{o}{=} \PY{n}{pd}\PY{o}{.}\PY{n}{read\PYZus{}csv}\PY{p}{(}\PY{l+s+s1}{\PYZsq{}}\PY{l+s+s1}{dataset/discretized\PYZus{}adult.test}\PY{l+s+s1}{\PYZsq{}}\PY{p}{)}
\end{Verbatim}
\end{tcolorbox}

    \hypertarget{ux7f16ux7801}{%
\subsection{编码}\label{ux7f16ux7801}}

为了方便表示,可以考虑将离散属性编码为整数。但在本例中是一个可选的步骤,直接用字符串表示的属性值表示属性取值同样可以,且具有更高的可读性\emph{(但可能略微损失少许性能,因为处理字符串比处理整数稍慢)}。

我省略了编码这一步骤,直接用属性字符串值表示节点内容。

    \hypertarget{ux6784ux5efaux51b3ux7b56ux6811}{%
\subsection{构建决策树}\label{ux6784ux5efaux51b3ux7b56ux6811}}

构建决策树的过程参考了``西瓜书''第 4 章图 4.2
的伪代码,并做了一些修改。修改了当样例为空时的行为,并增加了一个简单的剪枝条件。表示样例、属性的数据结构均使用
DataFrame。决策树表示为字典,字典的键由树节点、树边交替构成。

方便起见,我将训练好的决策树保存为 \texttt{tree\_id3.json} 文件。

    \begin{tcolorbox}[breakable, size=fbox, boxrule=1pt, pad at break*=1mm,colback=cellbackground, colframe=cellborder]
\prompt{In}{incolor}{8}{\hspace{4pt}}
\begin{Verbatim}[commandchars=\\\{\}]
\PY{c+c1}{\PYZsh{} 训练决策树}
\PY{k}{def} \PY{n+nf}{treeGenerate}\PY{p}{(}\PY{n}{df}\PY{p}{,} \PY{n}{mostImportant}\PY{p}{)}\PY{p}{:}
    \PY{l+s+sd}{\PYZdq{}\PYZdq{}\PYZdq{}生成一棵完整的决策树。}
\PY{l+s+sd}{    }
\PY{l+s+sd}{    Args:}
\PY{l+s+sd}{        df: 训练集,其中标签是名为\PYZsq{}50K\PYZsq{}的列。}
\PY{l+s+sd}{        mostImportant: 获得最优划分属性的函数。}
\PY{l+s+sd}{    Returns:}
\PY{l+s+sd}{        由字典表示的树。字典的键由树节点、树边交替构成;字典的值是子树或叶节点。}
\PY{l+s+sd}{    \PYZdq{}\PYZdq{}\PYZdq{}}
    \PY{c+c1}{\PYZsh{} 若所有样本属于同一类别,则返回该类别}
    \PY{k}{if} \PY{n+nb}{len}\PY{p}{(}\PY{n}{df}\PY{p}{[}\PY{l+s+s1}{\PYZsq{}}\PY{l+s+s1}{50K}\PY{l+s+s1}{\PYZsq{}}\PY{p}{]}\PY{o}{.}\PY{n}{unique}\PY{p}{(}\PY{p}{)}\PY{p}{)} \PY{o}{==} \PY{l+m+mi}{1}\PY{p}{:}
        \PY{k}{return} \PY{n}{df}\PY{p}{[}\PY{l+s+s1}{\PYZsq{}}\PY{l+s+s1}{50K}\PY{l+s+s1}{\PYZsq{}}\PY{p}{]}\PY{o}{.}\PY{n}{iloc}\PY{p}{[}\PY{l+m+mi}{0}\PY{p}{]}  \PY{c+c1}{\PYZsh{} 叶节点,返回标签}
    \PY{c+c1}{\PYZsh{} 若属性集为空,返回样本数最多的类}
    \PY{k}{if} \PY{n+nb}{len}\PY{p}{(}\PY{n}{df}\PY{o}{.}\PY{n}{columns}\PY{p}{)} \PY{o}{==} \PY{l+m+mi}{1}\PY{p}{:}
        \PY{k}{return} \PY{n}{df}\PY{p}{[}\PY{l+s+s1}{\PYZsq{}}\PY{l+s+s1}{50K}\PY{l+s+s1}{\PYZsq{}}\PY{p}{]}\PY{o}{.}\PY{n}{value\PYZus{}counts}\PY{p}{(}\PY{p}{)}\PY{o}{.}\PY{n}{index}\PY{p}{[}\PY{l+m+mi}{0}\PY{p}{]}  \PY{c+c1}{\PYZsh{} 叶节点,返回居右最多样本数的标签}
    \PY{k}{if} \PY{n+nb}{len}\PY{p}{(}\PY{n}{df}\PY{p}{)} \PY{o}{\PYZlt{}} \PY{l+m+mi}{200}\PY{p}{:}  \PY{c+c1}{\PYZsh{} 剪枝}
        \PY{k}{return} \PY{n}{df}\PY{p}{[}\PY{l+s+s1}{\PYZsq{}}\PY{l+s+s1}{50K}\PY{l+s+s1}{\PYZsq{}}\PY{p}{]}\PY{o}{.}\PY{n}{value\PYZus{}counts}\PY{p}{(}\PY{p}{)}\PY{o}{.}\PY{n}{index}\PY{p}{[}\PY{l+m+mi}{0}\PY{p}{]}

    \PY{n}{best\PYZus{}attribute} \PY{o}{=} \PY{n}{mostImportant}\PY{p}{(}\PY{n}{df}\PY{p}{)}  \PY{c+c1}{\PYZsh{} 最优划分属性}
    \PY{n}{tree} \PY{o}{=} \PY{p}{\PYZob{}}\PY{n}{best\PYZus{}attribute}\PY{p}{:} \PY{p}{\PYZob{}}\PY{p}{\PYZcb{}}\PY{p}{\PYZcb{}}         \PY{c+c1}{\PYZsh{} 准备构造当前节点}

    \PY{c+c1}{\PYZsh{} 对原始数据集中该属性的所有取值(注意这里用的不是子集,否则会导致树不完整)}
    \PY{k}{for} \PY{n}{value} \PY{o+ow}{in} \PY{n}{adult\PYZus{}data\PYZus{}df}\PY{p}{[}\PY{n}{best\PYZus{}attribute}\PY{p}{]}\PY{o}{.}\PY{n}{unique}\PY{p}{(}\PY{p}{)}\PY{p}{:}
        \PY{n}{next\PYZus{}df} \PY{o}{=} \PY{n}{df}\PY{p}{[}\PY{n}{df}\PY{p}{[}\PY{n}{best\PYZus{}attribute}\PY{p}{]}\PY{o}{==}\PY{n}{value}\PY{p}{]}\PY{o}{.}\PY{n}{drop}\PY{p}{(}\PY{p}{[}\PY{n}{best\PYZus{}attribute}\PY{p}{]}\PY{p}{,} \PY{n}{axis}\PY{o}{=}\PY{l+m+mi}{1}\PY{p}{)}
        \PY{k}{if} \PY{n+nb}{len}\PY{p}{(}\PY{n}{next\PYZus{}df}\PY{p}{)} \PY{o}{==} \PY{l+m+mi}{0}\PY{p}{:}  \PY{c+c1}{\PYZsh{} 该取值的样本集为空}
            \PY{n}{tree}\PY{p}{[}\PY{n}{best\PYZus{}attribute}\PY{p}{]}\PY{p}{[}\PY{n}{value}\PY{p}{]} \PY{o}{=}  \PY{n}{df}\PY{p}{[}\PY{l+s+s1}{\PYZsq{}}\PY{l+s+s1}{50K}\PY{l+s+s1}{\PYZsq{}}\PY{p}{]}\PY{o}{.}\PY{n}{value\PYZus{}counts}\PY{p}{(}\PY{p}{)}\PY{o}{.}\PY{n}{index}\PY{p}{[}\PY{l+m+mi}{0}\PY{p}{]}  \PY{c+c1}{\PYZsh{} 此处本应直接返回叶节点,但实验结果表明继续分枝效果更好,且对性能影响很小}
        \PY{k}{else}\PY{p}{:}  \PY{c+c1}{\PYZsh{} 递归}
            \PY{n}{tree}\PY{p}{[}\PY{n}{best\PYZus{}attribute}\PY{p}{]}\PY{p}{[}\PY{n}{value}\PY{p}{]} \PY{o}{=} \PY{n}{treeGenerate}\PY{p}{(}\PY{n}{next\PYZus{}df}\PY{p}{,} \PY{n}{mostImportant}\PY{p}{)}
    \PY{k}{return} \PY{n}{tree}  \PY{c+c1}{\PYZsh{} 返回子树}
\end{Verbatim}
\end{tcolorbox}

    \begin{tcolorbox}[breakable, size=fbox, boxrule=1pt, pad at break*=1mm,colback=cellbackground, colframe=cellborder]
\prompt{In}{incolor}{9}{\hspace{4pt}}
\begin{Verbatim}[commandchars=\\\{\}]
\PY{k}{def} \PY{n+nf}{id3}\PY{p}{(}\PY{n}{df}\PY{p}{)}\PY{p}{:}
    \PY{l+s+sd}{\PYZdq{}\PYZdq{}\PYZdq{}ID3算法划分属性。}
\PY{l+s+sd}{    Args:}
\PY{l+s+sd}{        df: 要进行属性划分的数据集,其中标签是名为\PYZsq{}50K\PYZsq{}的列。调用条件保证df至少有两列(包括标签列)。}
\PY{l+s+sd}{    Returns:}
\PY{l+s+sd}{        一个属性,按该属性划分可以使信息增益最大。}
\PY{l+s+sd}{    \PYZdq{}\PYZdq{}\PYZdq{}}
    \PY{n}{attributes} \PY{o}{=} \PY{n+nb}{list}\PY{p}{(}\PY{n}{df}\PY{o}{.}\PY{n}{columns}\PY{p}{)}
    \PY{n}{attributes}\PY{o}{.}\PY{n}{remove}\PY{p}{(}\PY{l+s+s1}{\PYZsq{}}\PY{l+s+s1}{50K}\PY{l+s+s1}{\PYZsq{}}\PY{p}{)}  \PY{c+c1}{\PYZsh{} 标签列不是属性,去除后attributes中至少有一个属性}

    \PY{n}{max\PYZus{}ig} \PY{o}{=} \PY{l+m+mi}{0}
    \PY{n}{best\PYZus{}attribute} \PY{o}{=} \PY{n}{attributes}\PY{p}{[}\PY{l+m+mi}{0}\PY{p}{]}
    \PY{k}{for} \PY{n}{attribute} \PY{o+ow}{in} \PY{n}{attributes}\PY{p}{[}\PY{l+m+mi}{1}\PY{p}{:}\PY{p}{]}\PY{p}{:}
        \PY{n}{current\PYZus{}ig} \PY{o}{=} \PY{n}{informationGain}\PY{p}{(}\PY{n}{df}\PY{p}{,} \PY{n}{attribute}\PY{p}{)}
        \PY{k}{if} \PY{n}{current\PYZus{}ig} \PY{o}{\PYZgt{}} \PY{n}{max\PYZus{}ig}\PY{p}{:}
            \PY{n}{best\PYZus{}attribute} \PY{o}{=} \PY{n}{attribute}
            \PY{n}{max\PYZus{}ig} \PY{o}{=} \PY{n}{current\PYZus{}ig}
    \PY{k}{return} \PY{n}{best\PYZus{}attribute}
\end{Verbatim}
\end{tcolorbox}

    \begin{tcolorbox}[breakable, size=fbox, boxrule=1pt, pad at break*=1mm,colback=cellbackground, colframe=cellborder]
\prompt{In}{incolor}{10}{\hspace{4pt}}
\begin{Verbatim}[commandchars=\\\{\}]
\PY{n}{tree\PYZus{}id3} \PY{o}{=} \PY{n}{treeGenerate}\PY{p}{(}\PY{n}{adult\PYZus{}data\PYZus{}df}\PY{p}{,} \PY{n}{id3}\PY{p}{)}
\end{Verbatim}
\end{tcolorbox}

    \begin{tcolorbox}[breakable, size=fbox, boxrule=1pt, pad at break*=1mm,colback=cellbackground, colframe=cellborder]
\prompt{In}{incolor}{11}{\hspace{4pt}}
\begin{Verbatim}[commandchars=\\\{\}]
\PY{c+c1}{\PYZsh{} 把决策树保存为JSON文件}
\PY{k}{with} \PY{n+nb}{open}\PY{p}{(}\PY{l+s+s1}{\PYZsq{}}\PY{l+s+s1}{tree\PYZus{}id3.json}\PY{l+s+s1}{\PYZsq{}}\PY{p}{,} \PY{l+s+s1}{\PYZsq{}}\PY{l+s+s1}{w}\PY{l+s+s1}{\PYZsq{}}\PY{p}{)} \PY{k}{as} \PY{n}{f}\PY{p}{:}
    \PY{n}{json}\PY{o}{.}\PY{n}{dump}\PY{p}{(}\PY{n}{tree\PYZus{}id3}\PY{p}{,} \PY{n}{f}\PY{p}{)} 
\end{Verbatim}
\end{tcolorbox}

    \hypertarget{ux9a8cux8bc1}{%
\subsection{验证}\label{ux9a8cux8bc1}}

在测试集上检验决策树模型的准确率。

    \begin{tcolorbox}[breakable, size=fbox, boxrule=1pt, pad at break*=1mm,colback=cellbackground, colframe=cellborder]
\prompt{In}{incolor}{12}{\hspace{4pt}}
\begin{Verbatim}[commandchars=\\\{\}]
\PY{k}{def} \PY{n+nf}{testSample}\PY{p}{(}\PY{n}{sample}\PY{p}{,} \PY{n}{tree}\PY{p}{)}\PY{p}{:}
    \PY{l+s+sd}{\PYZdq{}\PYZdq{}\PYZdq{}测试一个样本的正确性。}
\PY{l+s+sd}{    Args:}
\PY{l+s+sd}{        sample: 一个待测试样本。}
\PY{l+s+sd}{        tree: 决策树模型。}
\PY{l+s+sd}{    Returns:}
\PY{l+s+sd}{        True表示正确,False表示错误。}
\PY{l+s+sd}{    \PYZdq{}\PYZdq{}\PYZdq{}}
    \PY{k}{while} \PY{n+nb}{type}\PY{p}{(}\PY{n}{tree}\PY{p}{)} \PY{o}{==} \PY{n+nb}{type}\PY{p}{(}\PY{p}{\PYZob{}}\PY{p}{\PYZcb{}}\PY{p}{)}\PY{p}{:}  \PY{c+c1}{\PYZsh{} 子树类型一旦不是字典则表示到达叶节点}
        \PY{n}{attribute} \PY{o}{=} \PY{n+nb}{list}\PY{p}{(}\PY{n}{tree}\PY{o}{.}\PY{n}{keys}\PY{p}{(}\PY{p}{)}\PY{p}{)}\PY{p}{[}\PY{l+m+mi}{0}\PY{p}{]}
        \PY{n}{tree} \PY{o}{=} \PY{n}{tree}\PY{p}{[}\PY{n}{attribute}\PY{p}{]}\PY{p}{[}\PY{n}{sample}\PY{p}{[}\PY{n}{attribute}\PY{p}{]}\PY{p}{]}
    \PY{k}{return} \PY{n}{tree} \PY{o}{==} \PY{n}{sample}\PY{p}{[}\PY{l+s+s1}{\PYZsq{}}\PY{l+s+s1}{50K}\PY{l+s+s1}{\PYZsq{}}\PY{p}{]}

\PY{k}{def} \PY{n+nf}{test}\PY{p}{(}\PY{n}{df}\PY{p}{,} \PY{n}{tree}\PY{p}{)}\PY{p}{:}
    \PY{l+s+sd}{\PYZdq{}\PYZdq{}\PYZdq{}测试给定数据集上的预测正确率。}
\PY{l+s+sd}{    Args:}
\PY{l+s+sd}{        df: 测试数据集。}
\PY{l+s+sd}{        tree: 决策树模型。}
\PY{l+s+sd}{    Returns:}
\PY{l+s+sd}{        预测正确率。}
\PY{l+s+sd}{    \PYZdq{}\PYZdq{}\PYZdq{}}
    \PY{n}{correct\PYZus{}count} \PY{o}{=} \PY{l+m+mi}{0}
    \PY{k}{for} \PY{n}{i} \PY{o+ow}{in} \PY{n+nb}{range}\PY{p}{(}\PY{n+nb}{len}\PY{p}{(}\PY{n}{df}\PY{p}{)}\PY{p}{)}\PY{p}{:}
        \PY{k}{if} \PY{n}{testSample}\PY{p}{(}\PY{n}{df}\PY{o}{.}\PY{n}{iloc}\PY{p}{[}\PY{n}{i}\PY{p}{]}\PY{p}{,} \PY{n}{tree}\PY{p}{)}\PY{p}{:}
            \PY{n}{correct\PYZus{}count} \PY{o}{+}\PY{o}{=} \PY{l+m+mi}{1}
    \PY{k}{return} \PY{n}{correct\PYZus{}count} \PY{o}{/} \PY{n+nb}{len}\PY{p}{(}\PY{n}{df}\PY{p}{)}
\end{Verbatim}
\end{tcolorbox}

    \begin{tcolorbox}[breakable, size=fbox, boxrule=1pt, pad at break*=1mm,colback=cellbackground, colframe=cellborder]
\prompt{In}{incolor}{13}{\hspace{4pt}}
\begin{Verbatim}[commandchars=\\\{\}]
\PY{c+c1}{\PYZsh{} 从JSON文件中读取决策树}
\PY{k}{with} \PY{n+nb}{open}\PY{p}{(}\PY{l+s+s1}{\PYZsq{}}\PY{l+s+s1}{tree\PYZus{}id3.json}\PY{l+s+s1}{\PYZsq{}}\PY{p}{)} \PY{k}{as} \PY{n}{f}\PY{p}{:}
    \PY{n}{tree\PYZus{}id3} \PY{o}{=} \PY{n}{json}\PY{o}{.}\PY{n}{load}\PY{p}{(}\PY{n}{f}\PY{p}{)}
\end{Verbatim}
\end{tcolorbox}

    \begin{tcolorbox}[breakable, size=fbox, boxrule=1pt, pad at break*=1mm,colback=cellbackground, colframe=cellborder]
\prompt{In}{incolor}{14}{\hspace{4pt}}
\begin{Verbatim}[commandchars=\\\{\}]
\PY{c+c1}{\PYZsh{} 训练集准确率(供参考)}
\PY{n}{test}\PY{p}{(}\PY{n}{adult\PYZus{}data\PYZus{}df}\PY{p}{,} \PY{n}{tree\PYZus{}id3}\PY{p}{)}
\end{Verbatim}
\end{tcolorbox}

            \begin{tcolorbox}[breakable, boxrule=.5pt, size=fbox, pad at break*=1mm, opacityfill=0]
\prompt{Out}{outcolor}{14}{\hspace{3.5pt}}
\begin{Verbatim}[commandchars=\\\{\}]
0.8528914959614262
\end{Verbatim}
\end{tcolorbox}
        
    \begin{tcolorbox}[breakable, size=fbox, boxrule=1pt, pad at break*=1mm,colback=cellbackground, colframe=cellborder]
\prompt{In}{incolor}{15}{\hspace{4pt}}
\begin{Verbatim}[commandchars=\\\{\}]
\PY{c+c1}{\PYZsh{} 测试集准确率}
\PY{n}{test}\PY{p}{(}\PY{n}{adult\PYZus{}test\PYZus{}df}\PY{p}{,} \PY{n}{tree\PYZus{}id3}\PY{p}{)}
\end{Verbatim}
\end{tcolorbox}

            \begin{tcolorbox}[breakable, boxrule=.5pt, size=fbox, pad at break*=1mm, opacityfill=0]
\prompt{Out}{outcolor}{15}{\hspace{3.5pt}}
\begin{Verbatim}[commandchars=\\\{\}]
0.8469381487623611
\end{Verbatim}
\end{tcolorbox}
        
    最终,该决策树模型在测试集上的准确率为 84.7\%。


    % Add a bibliography block to the postdoc
    
    
    
    \end{document}
